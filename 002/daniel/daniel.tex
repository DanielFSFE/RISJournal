H: Wer bist du, was machst du gerade (Ausbildung) und wie alt bist du?


D: Hallo, mein Name ist Daniel Rittmannsberger und ich bin 15 Jahre alt. Im Moment besuche ich die htl donaustadt im 22. Bezirk.





H: Wie bist du zur FSFE und zu freier Software gekommen?


D: Das war als ich auf der GameCity war und zu deinem Stand ging, da ich eine Visistenkarte wollte, traf ich den FSFE-Aktivisten
   Franz Gratzer, da dieser am selben Stand die FSFE anpries. Er erklärte mir sofort voller Elan alles über freie Software und so
   unterhielten wir uns den restlichen Tag über freie Software. Bevor ich mit ihm gesprochen hab, hatte ich mich noch nie
   richtig dafür Interessiert aber er hatte so viel Motivaton, dass er es schaffte mich dafür zu begeistern. Das restliche
   Wochenende "Arbeitete" ich noch am Stand, da es mir sehr viel Spaß machte. Danach ging ich zu den FSFE Treffen und
   informierte mich mehr über freie Software.





H: Kennst du viele Jugendliche Deines Alters mit ähnlichen Interessen ?


D: Nein, niemanden und in meiner Klasse ist es auch schwer Leute für freie Software zu begeistern, da sie nur Videospiele
   spielen wollen und die gehen meistens nur auf Windows.





H: Bechreibe ein wenig Deinen Schultyp (Gymnasium / HTL). Inwiefern bist Du in Deiner Schule / seitens Deiner Lehrer 
   mit freier Software in Berührung gekommen? Was habt Ihr darüber gelernt? Was hast du dir selber beigebracht?


H: Wie war Dein erster Eindruck von FSFE ?



H: Wie wirkten die FSFE Aktivisten auf Dich ? Welche Nerd-Klischees (sofern du welche hattest) wurden bestätigt, 
   welche nicht ?


H: Kamst Du dir seltsam vor als einziger Jugendlicher / gelangweilt ?


H: Wie würdest Du die Ziele von FSFE beschreiben ?


H: Was war dein seltsamstes / coolstes / lustigstes Erlebnis im Zusammenhang mit FSFE bisher ?


H: Eine Message für junge Leser ?

